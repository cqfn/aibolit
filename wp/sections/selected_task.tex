%Here will be paragraph linking previous sections with "Code Readability".

%The research question is the following: is it possible, using static code analysis to do 
%a code refactoring recommendations to improve a code readability? 

We are looking for an answer to the following research question: 
Is it possible,
combining static code analysis and ML, 
to detect defects in a Java class and 
to give specific recommendations for its refactoring? 

To answer this question we are doing the following research:
\begin{enumerate*}[label={\alph*)}]
    \item We take a set of Java \emph{classes};
    \item We collect a number of static analysis \emph{metrics} per each class;
    \item We locate \emph{code patterns} inside each class;
    \item We ask volunteer programmers to review 
      each class to estimate their \emph{readability};
    \item We put collected data together into a \emph{dataset}; 
    \item We find \emph{relations} between metrics and patterns in the dataset; 
    \item We take an unseen Java class and \emph{locate code patterns}, which impact the quality more then others.
\end{enumerate*}
    
\textbf{Classes}.
We parse Java projects from GitHub to obtain training Java classes. 
All parsed projects must have more than 100 stars, and more than five collaborators. 
Collected Java classes must have more than 50 lines of code and less than 300.  

\textbf{Metrics}.
The code metrics are calculated using existing open-source
tools like CheckStyle and SourceMeter. Examples of metrics are Lines of Code (LoC), 
Cyclomatic Complexity (CC), and Number of Incoming Invocations (NII).

\textbf{Code Patterns}.
AST patterns are the features that reflect the presence 
of some syntax structures in the code, 
for example number of nested \texttt{FOR} loops 
or the amount of ternary operators. The AST patterns are manually designed 
and always point to a specific place in the code. 

\textbf{Readability}.
To gather readability characteristics we conduct
a survey, where volunteering programmers are asked to estimate
the readability of training Java classes, giving answers on a $[0..9]$ scale, 
where 0 means not readable at all and 9 means perfectly readable.
Each programmer is asked to review some snippets from 
the entire training dataset, which means that each
snippet is reviewed by a few programmers. The readability ``score'' per
snippet is a mean of all answers collected.

\textbf{Dataset}.
We combine and put the data into the single dataset. 
In the \autoref{tab:table1} you can see the example how dataset looks
(here CC, CBO, LCOM, and NMD are acronyms for software metrics, 
while RS is the Readability Score collected from volunteers).

\begin{table}[H]
\begin{center}
\begin{tabular}{rrrrrr}
\hline
CC & CBO & LCOM & NMD & \dots & RS \\
\hline
3 & 6 & 34 & 2 & \dots & 4.3 \\
4 & 5 & 55 & 1 & \dots & 2.4 \\
3 & 5 & 22 & 0 & \dots & 5.2 \\ 
\hline
\end{tabular}
\end{center}
\caption{Example of collected dataset}
\label{tab:table1}
\end{table}

\textbf{Relations}.
Using ML methods we are going to learn how code metrics, 
AST patterns and readability are related. 
We are planning to use ML techniques to investigate the importance of features 
with respect to the readability and find combinations that have stronger impact.

\textbf{Locate Code Patterns}.
Finally, for an unseen Java class we can calculate all features except the readability. 
Using known features and the knowledge about relations between features we are planning
to give refactoring recommendations and provide links to the lines of code 
where most problematic patterns are located.   

